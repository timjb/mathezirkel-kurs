\documentclass{article}

\usepackage[ngerman]{babel}
\usepackage[margin=0.7in]{geometry}
\usepackage[parfill]{parskip}
\usepackage[utf8]{inputenc}

\usepackage{amsmath,amssymb,amsfonts,amsthm}

\begin{document}

\section{Lösung zum Mathezirkelzettel vom 10. März 2017}

\textbf{Zu Aufgabe 1:}

Falls Max nur noch einen Wurf übrig hat, so ist sein erwarteter Gewinn gleich dem Durchschnitt aller möglichen Gewinne, also

\[ \tfrac{1}{6} \cdot (1 + 2 + 3 + 4 + 5 + 6) = 3,5 \]

Euro. Angenommen, Max würfelt das zweite Mal.

\begin{itemize}
  \item Falls Max weniger als $3,5$ würfelt, also eine 1, 2 oder 3, so sollte er noch einmal würfeln, da er dadurch einen höheren erwarteten Gewinn hat. Dieser Fall tritt mit einer Wahrscheinlichkeit von $\tfrac{3}{6} = \tfrac{1}{2}$ ein.
  \item Falls Max eine 4, 5 oder 6 würfelt, so sollte er diesen Betrag annehmen, da er über den 3,5 Euro liegt.
\end{itemize}

Sein erwarteter Gewinn beim zweiten Mal würfeln beträgt damit

\[ \tfrac{1}{2} \cdot 3,5 + \tfrac{1}{6} \cdot 4 + \tfrac{1}{6} \cdot 5 + \tfrac{1}{6} \cdot 6 = 4,25 \]

Euro. Angenommen, Max würfelt das erste Mal.

\begin{itemize}
  \item Falls Max weniger als $4,25$ würfelt, also eine 1, 2, 3 oder 4, so sollte er noch einmal würfeln, da er dadurch einen höheren erwarteten Gewinn hat. Dieser Fall tritt mit einer Wahrscheinlichkeit von $\tfrac{4}{6} = \tfrac{2}{3}$ ein.
  \item Falls Max eine 5 oder 6 würfelt, so sollte er diesen Betrag annehmen, da er über den 3,5 Euro liegt.
\end{itemize}

Sein erwarteter Gewinn am Anfang des Würfelspiels ist damit

\[ \tfrac{2}{3} \cdot 4,25 + \tfrac{1}{6} \cdot 5 + \tfrac{1}{6} \cdot 6 = 4 \tfrac{2}{3} \]

Euro.

\textbf{Zu Aufgabe 2:}

a) Rot darf solange würfeln, bis Rot keine Sechs, also eine Augenzahl zwischen 1 und 5 würfelt. Alle diese fünf Augenzahlen sind gleich wahrscheinlich. Damit beträgt die Wahrscheinlichkeit, dass Rot schlussendlich eine eins würfelt und damit einräumt genau~$\tfrac{1}{5}$.

b) Wir rechnen wieder rückwärts:

\begin{enumerate}
  \item[a)] Steht Grün direkt hinter Rot und ist Grün am Zug, so ist die Wahrscheinlichkeit, dass Grün Rot noch schlägt gleich der Wahrscheinlichkeit, dass Grün jetzt eine Eins würfelt, also gleich~$\tfrac{1}{6}$.
  \item[b)] Steht Grün direkt hinter Rot und ist Rot am Zug, so ist die Wahrscheinlichkeit, dass Grün Rot noch schlägt gleich der Wahrscheinlichkeit, dass Rot \emph{nicht} einräumt multipliziert mit der Wahrscheinlichkeit aus Situation a), also $\tfrac{4}{5} \cdot \tfrac{1}{6} = \tfrac{2}{15}$.
  \item[c)]
    Steht Grün zwei Felder hinter Rot und ist Grün am Zug, so gibt es beim nächsten Würfeln mehrere Möglichkeiten:
    \begin{itemize}
      \item Grün würfelt eine Zwei und schlägt damit Rot sofort. Dieser Fall hat eine Wahrscheinlichkeit von~$\tfrac{1}{6}$
      \item Grün würfelt eine Eins und rückt damit auf, sodass die grüne Figur direkt hinter der roten steht. Dieser Fall hat ebenso eine Wahrscheinlichkeit von~$\tfrac{1}{6}$. Die Wahrscheinlichkeit, dass in diesem Fall es Grün noch gelingt, Rot zu schlagen, wurde in Situation b ausgerechnet.
      \item In allen anderen Fällen muss Grün an Rot vorbeiziehen.
    \end{itemize}
    Damit ist die Wahrscheinlichkeit, dass Grün noch Rot schlagen kann, gleich
    \[ \tfrac{1}{6} + \tfrac{1}{6} \cdot \tfrac{2}{15} = \tfrac{1}{6} \cdot (1 + \tfrac{2}{15}) = \tfrac{1}{6} \cdot \tfrac{17}{15} = \tfrac{17}{90}. \]
  \item[d)] Steht Grün zwei Felder hinter Rot und ist Rot am Zug, so ist die Wahrscheinlichkeit, dass Grün Rot noch schlägt gleich der Wahrscheinlichkeit, dass Rot \emph{nicht} einräumt multipliziert mit der Wahrscheinlichkeit aus Situation c), also $\tfrac{4}{5} \cdot \tfrac{17}{90} = \tfrac{34}{225}$.
  \item[e)]
    Steht Grün drei Felder hinter Rot und ist Grün am Zug, so gibt es beim nächsten Würfeln mehrere Möglichkeiten, die jeweils eine Wahrscheinlichkeit von~$\tfrac{1}{6}$, wie Grün noch Rot schlagen kann:
    \begin{itemize}
      \item Grün würfelt eine Eins und rückt damit bis auf zwei Felder hinter Rot auf. Die Wahrscheinlichkeit, dass in diesem Fall es Grün noch gelingt, Rot zu schlagen, wurde in Situation b) ausgerechnet.
      \item Grün würfelt eine zwei, sodass die grüne Figur direkt hinter der roten steht. Die Wahrscheinlichkeit, dass in diesem Fall es Grün noch gelingt, Rot zu schlagen, wurde in Situation d) ausgerechnet.
      \item Grün würfelt eine Drei und schlägt damit Rot sofort.
    \end{itemize}
    Somit ist die Wahrscheinlichkeit, dass Grün noch Rot schlagen kann, gleich
    \[
      \frac{1}{6} + \tfrac{1}{6} \cdot \tfrac{2}{15} + \tfrac{1}{6} \cdot \tfrac{34}{225} =
      \tfrac{1}{6} \cdot (1 + \tfrac{2}{15} + \tfrac{34}{225}) =
      \tfrac{1}{6} \cdot (\tfrac{225}{225} + \tfrac{30}{225} + \tfrac{34}{225}) =
      \tfrac{1}{6} \cdot \tfrac{289}{225} = \tfrac{289}{1350} \approx 0,21407
    \]
\end{enumerate}

\textbf{Zu Aufgabe 3:}

Sei $x$ die erwartete Anzahl Würfe bis zu einer Sechs.

Wenn man eine Sechs würfeln will, dann würfelt man zunächst ein Mal:

\begin{itemize}
  \item Mit einer Wahrscheinlichkeit von~$\tfrac{1}{6}$ würfelt man direkt eine Sechs. In diesem Fall muss man keinen weiteren Wurf machen.
  \item Mit einer Wahrscheinlichkeit von~$\tfrac{5}{6}$ würfelt man keine Sechs. In diesem Fall muss man noch erwartungsgemäß~$x$ weitere Würfe machen.
\end{itemize}

Daraus ergibt sich die Gleichung

\[ x = 1 + \tfrac{1}{6} \cdot 0 + \tfrac{5}{6} \cdot x. \]

Es folgt, dass $x = 6$.

\textbf{Zu Aufgabe 4:}

Wir bezeichnen mit

\begin{itemize}
  \item Situation 1: $a$ die erwartete Anzahl Münzwürfe bis man zweimal Zahl hintereinander geworfen hat unter der Voraussetzung, dass man noch keinen Wurf gemacht hat oder dass man beim letzten Wurf Kopf geworfen hat.
  \item Situation 2: $b$ die erwartete Anzahl Münzwürfe bis man zweimal Zahl hintereinander geworfen wenn man beim vorhergehenden aber nicht beim Wurf davor Zahl geworfen hat.
\end{itemize}

Würfelt man in Situation~1 Zahl, so ist man vor dem nächsten Würfeln in Situation 2 und man muss noch erwartungsgemäß~$b$ Mal würfeln. Wenn man in Situation~1 Kopf würfelt, so bleibt man in Situation~1 und man muss erwartungsgemäß noch weitere~$a$ Mal die Münze werfen. Als Gleichung formuliert:

\[ a = 1 + \tfrac{1}{2} \cdot a + \tfrac{1}{2} \cdot b. \]

Würfelt man in Situation~2 Zahl, so ist man fertig und muss nicht mehr würfeln. Wenn man in Situation~2 Kopf würfelt, so landet man wieder in Situation~1 und man muss noch erwartete weitere~$a$ Mal würfeln. Daraus ergibt sich die Gleichung

\[ b = 1 + \tfrac{1}{2} \cdot 0 + \tfrac{1}{2} \cdot a = 1 + \tfrac{1}{2} \cdot a. \]

Setzt man die zweite Gleichung in die erste ein, so erhält man

\[ a = 1 + \tfrac{1}{2} \cdot a + \tfrac{1}{2} \cdot (1 + \tfrac{1}{2} \cdot a) = \tfrac{3}{2} + \tfrac{3}{4} \cdot a \]

Es folgt $a = 6$ und $b = 1 + \tfrac{1}{2} \cdot a = 1 + 3 = 4$.

\textbf{Zu Aufgabe 5:}

Wir bezeichnen mit

\begin{itemize}
  \item $a$ die erwartete Anzahl Schachteln, die ich noch kaufen muss, wenn ich noch kein MT habe.
  \item $b$ die erwartete Anzahl Schachteln, die ich noch kaufen muss, wenn ein MT habe.
  \item $c$ die erwartete Anzahl Schachteln, die ich noch kaufen muss, wenn ich schon zwei MTe habe.
\end{itemize}

Wenn man noch kein MT hat, so muss man sich erstmal eine Packung Müsli kaufen und hat danach ein MT:

\[ a = 1 + b. \]

Wenn man schon ein MT hat und sich eine Packung Müsli kauft, so hat man danach mit einer Wahrscheinlichkeit von~$\tfrac{2}{3}$ danach zwei MT und man muss sich noch erwartete $c$ Schachteln kaufen.
Mit einer Wahrscheinlichkeit von~$\tfrac{1}{3}$ bekommt man jedoch das MT, das man schon besitzt, und man muss noch erwartete~$b$ Schachteln kaufen.
Als Gleichung ausgedrückt:

\[ b = 1 + \tfrac{2}{3} \cdot c + \tfrac{1}{3} \cdot c \]

Wenn man schon zwei MTe hat und sich eine Packung kauft, so ist mit einer Wahrscheinlichkeit von~$\tfrac{1}{3}$ darin das noch fehlende MT und man muss sich keine neuen Packungen kaufen.
Mit einer Wahrscheinlichkeit von~$\tfrac{2}{3}$ bekommt man aber ein MT, das man schon besitzt. In dem Fall muss man noch erwartete~$c$ Schachteln kaufenchteln kaufen.

\[ c = 1 + \tfrac{1}{3} \cdot 0 + \tfrac{2}{3} \cdot c \]

Es folgt nacheinander: $c = 3$, $b = 4,5$, $c = 5,5$


\textbf{Zu Aufgabe 6:}

Seien $a$, $b$ bzw. $c$ die erwartete Anzahl an Schritten, wenn die Spinne drei, zwei bzw. eine Kante vom Käfer entfernt ist.

Befindet sich die Spinne an der gegenüberliegenden Ecke zum Marienkäfer (also drei Kanten entfernt), so führen alle drei Kanten zu einer Ecke, die nur noch zwei Kanten vom Käfer entfernt ist. Also:

\[ a = 1 + b \]

Befindet sich die Spinne an einer Ecke, die zwei Kanten vom Käfer entfernt ist, so führen, wie man im Bild sehen kann, zwei Kanten zu einer Ecke mit Distanz eins und eine Kante zur Ecke mit Distanz drei. Es folgt:

\[ b = 1 + \tfrac{2}{3} \cdot c + \tfrac{1}{3} \cdot a \]

Befindet sich die Spinne an einer Ecke mit Distanz eins vom Käfer, so führt eine der drei Kanten zum Käfer und die beiden anderen zu einer Ecke der Distanz zwei. Somit:

\[ c = 1 + \tfrac{1}{3} \cdot 0 + \tfrac{2}{3} \cdot b = 1 + \tfrac{2}{3} \cdot b \]

Setzt man die erste und die dritte Gleichung in die zweite Gleichung ein, so erhält man

\[
  b =
  1 + \tfrac{2}{3} \cdot (1 + \tfrac{2}{3} \cdot b) + \tfrac{1}{3} \cdot (1 + b) =
  2 + (\tfrac{2}{3} \cdot \tfrac{2}{3} + \tfrac{1}{3}) \cdot b =
  2 + (\tfrac{4}{9} + \tfrac{3}{9}) \cdot b
  2 + \tfrac{7}{9} \cdot b
\]

Es folgt $\tfrac{2}{9} \cdot b = 2$, also $b = 9$. Somit $a = 1 + b = 10$ und $c = 1 + \tfrac{2}{3} \cdot b = 7$.


\textbf{Zu Aufgabe 7:}

Folgende Situtation können während des Duells auftreten:

\begin{tabular}{r|l|l}
  & Notation & Beschreibung \\ \hline
  R & A $\rightarrowtail$ B; C & alle drei am Leben, A schießt auf B \\
  S & B $\rightarrowtail$ C; A & alle drei am Leben, B schießt auf C \\
  T & C $\rightarrowtail$ A; B & alle drei am Leben, C schießt auf A \\
  U & B $\rightarrowtail$ C & A tot, B schießt auf C \\
  V & C $\rightarrowtail$ B & A tot, C schießt auf B \\
  W & C $\rightarrowtail$ A & B tot, C schießt auf A \\
  X & A $\rightarrowtail$ C & B tot, A schießt auf C \\
  Y & A $\rightarrowtail$ B & C tot, A schießt auf B \\
  Z & B $\rightarrowtail$ A & C tot, B schießt auf A
\end{tabular}

a) Seien $p_R, p_S, \ldots, p_Z$ die Wahrscheinlichkeiten, dass A in der jeweiligen Situation gewinnt. Es gilt:

\begin{align*}
  p_R & = \tfrac{1}{2} p_W + \tfrac{1}{2} p_S \\
  p_S & = \tfrac{3}{4} p_X + \tfrac{1}{4} p_T \\
  p_T & = \tfrac{1}{2} p_U + \tfrac{1}{2} p_R \\
  p_U & = 0 \\
  p_V & = 0 \\
  p_W & = \tfrac{1}{2} p_X \\
  p_X & = \tfrac{1}{2} + \tfrac{1}{2} p_W \\
  p_Y & = \tfrac{1}{2} + \tfrac{1}{2} p_Z \\
  p_Z & = \tfrac{1}{4} p_Y \\
\end{align*}

Dieses Gleichungssystem hat die Lösung

\[
  p_R = \tfrac{128}{315}, \enspace
  p_S = \tfrac{443}{630}, \enspace
  p_T = \tfrac{64}{315}, \enspace
  p_W = \tfrac{1}{3}, \enspace
  p_X = \tfrac{2}{3}, \enspace
  p_Y = \tfrac{4}{7}, \enspace
  p_Z = \tfrac{1}{7}
\]

Somit gewinnt A mit einer Wahrscheinlichkeit von $\tfrac{128}{315} \approx 40,635 \%$.

b) Seien $n_R, n_S, \ldots, n_Z$ die in der jeweiligen Situation erwarteten Anahl Schüsse bis zum Ende des Triells. Es gilt:

\begin{align*}
  n_R & = 1 + \tfrac{1}{2} n_W + \tfrac{1}{2} n_S \\
  n_S & = 1 + \tfrac{3}{4} n_X + \tfrac{1}{4} n_T \\
  n_T & = 1 + \tfrac{1}{2} n_U + \tfrac{1}{2} n_R \\
  n_U & = 1 + \tfrac{1}{4} n_V \\
  n_V & = 1 + \tfrac{1}{2} n_U \\
  n_W & = 1 + \tfrac{1}{2} n_X \\
  n_X & = 1 + \tfrac{1}{2} n_W \\
  n_Y & = 1 + \tfrac{1}{2} n_Z \\
  n_Z & = 1 + \tfrac{1}{4} n_Y \\
\end{align*}

Die Lösung dieses Gleichungssystems ist

\[
  n_R = \tfrac{376}{105}, \enspace
  n_S = \tfrac{499}{210}, \enspace
  n_T = \tfrac{368}{105}, \enspace
  n_U = \tfrac{10}{7}, \enspace
  n_V = \tfrac{12}{7}, \enspace
  n_W = 2, \enspace
  n_X = 2, \enspace
  n_Y = \tfrac{12}{7}, \enspace
  n_Z = \tfrac{10}{7}
\]

Es werden also erwartete $\tfrac{376}{105} \approx 3,581$ Kugeln verschossen.

\textbf{Zu Aufgabe 8:}

Sei $x$ der Betrag, der es mir Wert ist, mit dem Automaten zu spielen.

Wenn ich spiele und den angebotenen Betrag ablehne, so darf ich erst in zwei Monaten wieder spielen.
Dieses Spiel wird mir in drei Monaten $x$ Euro wert sein.
Da ich den Gewinn aber erst frühestens in zwei Monaten bekomme, ist mir dieses Spiel aber jetzt nur $0,9 x$ Euro wert.

Meine Strategie sollte die folgende sein:

\begin{itemize}
  \item Falls ich beim Spiel mindestens $0,9 x$ Euro gewinne, dann sollte ich den Betrag annehmen, da er mehr ist, als mir das Spiel in zwei Monaten wert ist. Dieser Fall tritt mit einer Wahrscheinlichkeit von $\tfrac{100 - 0,9 x}{100}$ ein. Der durchschnittliche Gewinn in diesem Fall ist $\tfrac{100 + 0,9 x}{2}$.
  \item Falls ich beim Spiel weniger als $0,9 x$ Euro gewinne, dann sollte ich in zwei Monaten noch einmal spielen. Dieser Fall tritt mit einer Wahrscheinlichkeit von $\tfrac{0,9 x}{100}$ ein. Das Spiel in zwei Monaten ist mir noch $0,9 x$ Euro Wert.
\end{itemize}

Daraus folgt die Gleichung

\[ x = \frac{100 - 0,9 \cdot x}{100} \cdot \frac{100 + 0,9 \cdot x}{2} + \frac{0,9 \cdot x}{100} \cdot 0,9 \cdot x. \]

Vereinfacht man diese Gleichung und stellt die Terme um, so kommt man auf die quadratische Gleichung

\[ 0 = \tfrac{0,81}{200} x^2 - 1 x + 50. \]

Mit der Mitternachtsformel lassen sich die beiden Lösungen dieser Gleichung bestimmen:

\[
  x_1 = 69,6432, \quad
  x_2 = 177,27.
\]

Die zweite Lösung ist offensichtlich Quatsch, da man mit dem Automaten höchstens 100 Euro gewinnen kann.
Also ist $69$ Euro die richtige Antwort.

\textbf{Zu Aufgabe 9:}

a) Sei $x$ die Wahrscheinlichkeit, dass die von einem Bakterium begründete Linie irgendeinmal ausstirbt. Mit einer Wahrscheinlichkeit von~$\tfrac{1}{4}$ stirbt das Bakterium und hinterlässt gar keine Nachkommen. Andernfalls gibt es zwei direkte Nachkommen. Die Wahrscheinlichkeit, dass die Linie ausstirbt ist die Wahrscheinlichkeit, dass die Linien von beiden direkten Nachkommen aussterben, und somit (wegen der Unabhängigkeit der Überlebenswahrscheinlichkeit) gleich~$x^2$.

Es folgt:

\[ x = \tfrac{1}{4} + \tfrac{3}{4} x^2 \]

Diese Gleichung hat zwei Lösungen, $x_1 = \tfrac{1}{3}$ und $x_2 = 1$. Welche davon ist die richtige?

Um das herauszufinden, betrachten wir die Funktion

\[ f(z) = \tfrac{1}{4} + \tfrac{3}{4} z^2 \]

Es gilt: Ist $z$ die Wahrscheinlichkeit, dass die Linie eines Bakteriums innerhalb von $n$~Generationen ausstirbt, so ist $f(x)$ die Wahrscheinlichkeit, dass die Linie eines Bakteriums innerhalb von $(n+1)$ Generationen ausstirbt.

Folglich ist $z_m := f(f(\cdots(f(0))))$ (genau $m$ mal $f$) die Wahrscheinlichkeit, dass die Linie eines Bakteriums innerhalb von~$m$ Generationen ausstirbt.
Wir interessieren uns dafür, gegen welchen Wert sich diese Zahlen für große Werte von $m$ annähern. (Denn die Wahrscheinlichkeit, dass die Linie eines Bakteriums innerhalb von einer Million Generationen ausstirbt ist fast gleich der Wahrscheinlichkeit, dass sie irgendwann ausstirbt.)

Mit einem Computer können wir diese Zahlen berechnen:

\begin{tabular}{r|l}
  $z_0$ & 0,0 \\
  $z_1$ & 0,25 \\
  $z_2$ & 0,296875 \\
  $z_3$ & 0,31610107421875 \\
  $z_4$ & 0,32493991684168577 \\
  $z_5$ & 0,32918946216781125 \\
  $z_6$ & 0,3312742765017496 \\
  $z_7$ & 0,33230698470381825 \\
  $z_8$ & 0,33282094906220777 \\
  $z_9$ & 0,33307733810100154 \\
  & $\vdots$ \\
  $z_{100}$ & 0,3333333333333333
\end{tabular}

Somit ist $\tfrac{1}{3}$ die richtige Antwort.

b) Analog zu a) suchen wir die Lösung der Gleichung

\[ x = 0,6 + 0,4 x^2. \]

Nach der Mitternachtsformel hat diese Gleichung die Lösungen $x_1 = 1$ und $x_2 = 1,5$. Die zweite Lösung ist keine Wahrscheinlichkeit, da sie größer als eins ist. Somit ist eins die richtige Antwort.

Das heißt, die Wahrscheinlichkeit, dass die Bakterienkultur ausstirbt, ist $100\%$.

\end{document}