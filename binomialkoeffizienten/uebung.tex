\documentclass{uebungszettel}

% Für die Verwendung von `diagrams` mit LaTeX
\usepackage[backend=cairo, outputdir=diagrams]{diagrams-latex}
\usepackage{graphicx}

\begin{document}
\pagestyle{empty}

\maketitle{Klasse 8/9}{20. Januar 2017}

\begin{aufgabe}{Symmetrie der Binomialkoeffizienten}
  Es gilt ${n \choose k} = {n \choose n-k}$.
  Kannst du dies anhand des folgenden Diagramms beweisen?
  
\begin{diagram}
import Diagrams.BinomialCoefficients
import qualified NiceColours as NC

dia =
  bcSymmetryDiagram
    (ColourScheme { dotColour = black, selectionColour = NC.blue })
    (ColourScheme { dotColour = black, selectionColour = NC.green })
    5 3
\end{diagram}
\end{aufgabe}

\begin{aufgabe}{Eine Rekursionsformel mit Addition}
  Für $n \geq 1$ und $k \geq 1$ gilt ${n \choose k} = {n-1 \choose k-1} + {n-1 \choose k}$.
  Kannst du dies anhand des folgenden Diagramms (das den Fall $n = 6$ und $k = 4$ zeigt) beweisen?
  
\begin{diagram}
import Diagrams.BinomialCoefficients
import qualified NiceColours as NC

dia =
  bcAdditionIdentityDiagram
    (ColourScheme { dotColour = black, selectionColour = NC.teal })
    6 4
\end{diagram}
\end{aufgabe}

\begin{aufgabe}{Das Pascalsche Dreieck}
  TODO
\end{aufgabe}

\begin{aufgabe}{Eine Rekursionsformel mit Multiplikation}
  Für $n \geq 1$ und $k \geq 1$ gilt $k \cdot {n \choose k} = n \cdot {n-1 \choose k-1}$.
  Das folgende Diagram soll dies im Fall $n=5$ und $k=3$ zeigen: 
  $3 \cdot {5 \choose 3} = 5 \cdot {4 \choose 2}$.
  Kannst du den Beweis erklären?
  
\begin{diagram}
import Diagrams.BinomialCoefficients
import qualified NiceColours as NC

dia =
  bcMultiplicativeIdentityDiagram
    (ColourScheme { dotColour = black, selectionColour = NC.red })
    (ColourScheme { dotColour = NC.yellow, selectionColour = NC.red })
    5 3
\end{diagram}
\end{aufgabe}

\begin{aufgabe}{Summe verschobener Binomialkoeffizienten}
  Für alle $n \geq 0$ und $m \geq 0$ gilt die Gleichung
  \[
    {n+m+1 \choose n+1} = {n \choose n} + {n+1 \choose n} + \ldots + {n+m \choose n-1} + {n+m \choose n}.
  \]
  Erkläre diese Gleichung mit der Skizze, die den Fall $n=2$ und $m=3$ zeigt!
  
\begin{diagram}
import Diagrams.BinomialCoefficients
import qualified NiceColours as NC

dia =
  shiftedBcsIdentityDiagram
    (simpleColourScheme NC.orange)
    2 3
\end{diagram}
\end{aufgabe}

\begin{aufgabe}{Vandermondesche Identität}
  Die \emph{Vandermondesche Identität} besagt, dass für alle~$m$, $n$ und~$k$ gilt:
  \[
    {m+n \choose k} = {m \choose 0} \cdot {n \choose k} + {m \choose 1} \cdot {n \choose k-1} + \ldots + {m \choose k-1} \cdot {n \choose 1} + {m \choose k} \cdot {n \choose 0}
  \]
  Für $m=4$, $n=3$ und $k=2$ gilt also
  \[
    {4+3 \choose 2} = {4 \choose 0} \cdot {3 \choose 2} + {4 \choose 1} \cdot {3 \choose 1} + {4 \choose 2} \cdot {3 \choose 1}.
  \]
  Warum gilt diese Gleichung?
  Das kannst du anhand der Abbildung für den Fall $m = 3$, $n = 4$, $k = 3$ nachvollziehen:

\begin{diagram}
import Diagrams.BinomialCoefficients
import qualified NiceColours as NC

dia =
  vandermondIdentityDiagram schemeM schemeN 3 4 3
  where
    schemeM = ColourScheme { dotColour = darkgreen, selectionColour = NC.green }
    schemeN = ColourScheme { dotColour = darkred, selectionColour = NC.red }
\end{diagram}
\end{aufgabe}

\begin{aufgabe}{Alternierende Summe von Binomialkoeffizienten}
  Für alle natürlichen Zahlen~$n$ gilt:
  \[
    {n \choose 0} - {n \choose 1} + {n \choose 2} - \ldots \pm {n \choose n-1} \mp {n \choose n} = 0
  \]
  Dabei addiert man alle Zahlen ${n \choose k}$ mit~$k$ gerade und subtrahiert von dieser Summe alle Zahlen ${n \choose k}$ mit $k$~ungerade.
  Erkläre folgenden Beweis ohne Worte für diese Gleichung!

\begin{center}
\begin{diagram}
import Diagrams.BinomialCoefficients
import qualified NiceColours as NC

dia =
  alternatingBcsIdentityDiagram (simpleColourScheme olive) 6
    # center
    # rotateBy (-0.25)
    # scale 0.8
\end{diagram}
\end{center}

{\footnotesize Tipp: Wie viele Fünfer-Cliquen sind in jeder Zeile? Wann sind zwei Cliquen verbunden? Es kann hilfreich sein, die fünf Kreise jeder Clique durchzunummerieren von eins bis fünf, angefangen beim obersten Kreis.}
\end{aufgabe}

\begin{aufgabe}{Wege im Gitter}
  TODO
\end{aufgabe}

\end{document}